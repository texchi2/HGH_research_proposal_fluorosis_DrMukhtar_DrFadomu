\documentclass{article}
\usepackage{graphicx} % Required for inserting images
\usepackage[utf8]{inputenc}

%% move to titlepage
\title{Dental Fluorosis and Skeletal Fluorosis in Somaliland Population}
%\subtitle{Evaluating the Diagnostic Accuracy of CT-Scans and Serum Fluoride Levels in Detecting Skeletal Fluorosis in the Somaliland Population with Dental Fluorosis}
\author{Dr. Mukhtar , Dr. Fadumo, Mr. Hashi, and Dr. Tex}
\date{\today}

\usepackage{outlines}
\usepackage{svg}
\usepackage[hidelinks]{hyperref} % removal of green square
\usepackage{tikz}
\usetikzlibrary{positioning}

\usepackage[numbers]{natbib} % Use natbib instead of cite
% [numbers]


\begin{filecontents}{inline.bib}
@article{fluorosis,
  title={Skeletal fluorosis},
  author={},
  journal={Medical Dictionary},
  url={https://medical-dictionary.thefreedictionary.com/skeletal+fluorosis}
}

@article{springer,
  title={Skeletal Fluorosis},
  author={Subramanian Shankar and Vivek Vasdev},
  journal={SpringerLink},
  year={2022},
  url={https://link.springer.com/chapter/10.1007/978-3-031-05002-2_10}
}

@article{radiopaedia,
  title={Fluorosis},
  author={Henry Knipe},
  journal={Radiopaedia},
  year={2023},
  url={https://radiopaedia.org/articles/fluorosis}
}

@article{medindia,
  title={Fluorosis - Causes, Symptoms, Diagnosis, Treatment, Prevention},
  author={},
  journal={Medindia},
  year={2022},
  url={https://www.medindia.net/patients/patientinfo/fluorosis.htm}
}

@article{sellami,
    title = {Skeletal fluorosis: don’t miss the diagnosis!},
    author = {Sellami, Meriem and Riahi, Hend and Maatallah, Kaouther and Ferjani, Hanen and Chelli Bouaziz, Mouna and Ladeb, Mohamed Fethi},
    journal = {Skeletal Radiology},
    volume = {49},
    number = {3},
    pages = {345--357},
    year = {2020},
    url = {https://doi.org/10.1007/s00256-019-03302-0}
}

@article{contamination,
  title={Fluoride contamination, consequences and removal techniques in water: a review},
  author={Shaz Ahmad and Reena Singh and Tanvir Arfin},
  journal={Environmental Science: Advances (RSC Publishing)},
  year={2022},
  url={https://pubs.rsc.org/en/content/articlehtml/2022/va/d1va00039j}
}

\end{filecontents}

\begin{document}
%%


\begin{titlepage}
    \centering
    \vspace*{\fill}
    {\LARGE Dental Fluorosis and Skeletal Fluorosis in Somaliland Population\par}
    \vspace{1cm}
    {\large Dr. Mukhtar , Dr. Fadumo, Mr. Hashi, and Dr. Tex \\
    \date{\today}}
    \vspace*{\fill}
\end{titlepage}



%%
\begin{abstract}
\textbf{Background}. Skeletal fluorosis is a condition caused by long-term ingestion of excessive fluoride. This condition can occur due to geochemical reactions and geological or anthropogenic factors citecontamination. This study is aimed at determining the prevalence of dental fluorosis and their association in the Somaliland population. 

\textbf{Methods}. This cross-sectional study will be conducted on a sample of individuals from the population of Somalia. We will use computed tomography (CT) scans and serum fluorides levels to diagnose skeletal fluoriosis in participants with dental fluonosis. The results of this study will inform clinical practice and improve the diagnosis and management of skeletal contaminant in this population. 

\textbf{Results}. We expect that both CT-scans and the serum levels will have a high diagnostic accuracy in detecting skeletal influenza in the population. We also anticipate that our results will reveal an association between dental fluarosis and skeletal pneumonia in this group. 

\textbf{Conclusions}. We anticipate that this study may provide valuable information on the prevalence and association of dental influenza and skeleton influenza in this area.
\end{abstract}

%\maketitle

\section{Introduction}
Skeletal fluorosis is a condition caused by long-term ingestion of excessive fluoride. It is characterized by skeletal changes such as hyperostosis, osteopetrosis, and osteoporosis \cite{fluorosis}. Dental fluorosis, on the other hand, is characterized by hypoplasia of the dental enamel, which can result in mottled enamel \cite{fluorosis}. The presence of dental fluorosis can help in making the diagnosis of skeletal fluorosis \cite{springer}. Fluoride contamination in drinking water is a global issue, and it can occur due to geochemical reactions and geological or anthropogenic factors \cite{contamination}. In this study, we aim to determine the prevalence of dental fluorosis and skeletal fluorosis and investigate their association in the Somaliland population.

% vitamin D
% bone pain


\section{Literature Review}
Previous studies have shown that CT-scans can be used to detect changes in bone associated with fluorosis. Described features include increased bone density, osteosclerosis, trabecular blurring or haziness, compact bone thickening, periosteal bone formation, and ossification of the attachments of tendons, ligaments, and muscles \cite{radiopaedia}. Serum fluoride levels have also been used as a diagnostic tool for skeletal fluorosis \cite{medindia}. A study by Sellami et al. also provides a comprehensive review of skeletal fluorosis, including its diagnosis and management \cite{sellami}.
However, there is limited research on the use of these methods specifically in the Somaliland population.

\section{Specific Aims}
The specific aims of this study are to:
%\begin{enumerate}
%    \item Evaluate the diagnostic accuracy of CT-scans in detecting skeletal fluorosis in the Somaliland population with dental fluorosis.
%    \item Evaluate the diagnostic accuracy of serum fluoride levels in detecting skeletal fluorosis in the Somaliland population with dental fluorosis.
%    \item Compare the diagnostic accuracy of CT-scans and serum fluoride levels in detecting skeletal fluorosis in the Somaliland population with dental fluorosis.
%\end{enumerate}

\begin{outline} 
\1 The primary aim of this study is to determine the prevalence of dental fluorosis and skeletal fluorosis in the Somaliland population. 
\1 We will use CT-scans and serum fluoride levels to diagnose skeletal fluorosis in participants with dental fluorosis. 
\1 We will also investigate the association between dental fluorosis and skeletal fluorosis in the Somaliland population. 
\end{outline}


\section{Material and Method}
This study will be a cross-sectional study conducted on a sample of individuals from the Somaliland population with dental fluorosis. Participants will undergo a CT-scan and have their serum fluoride levels measured. The presence or absence of skeletal fluorosis will be determined based on established diagnostic criteria. The diagnostic accuracy of CT-scans and serum fluoride levels will be evaluated using sensitivity, specificity, positive predictive value, and negative predictive value.


%%% old
\begin{outline}

\1 Obtain the institution review board (IRB) approval for the project from the relevant institution. 
\1 Identify and recruit participants from the Somaliland population with dental fluorosis. Ensure that participants provide informed consent before enrolling in the study. 
\1 Collect data from participants, including demographics, medical history, CT-scan results, and serum fluoride levels. Ensure to follow all relevant data protection and privacy regulations when handling patient data. 
    \2 Schedule appointments for participants to undergo a CT-scan and have their serum fluoride levels measured. Provide participants with instructions on how to prepare for these tests, if necessary. 
    \2 During the appointment, collect demographic information from participants, such as age, sex, and ethnicity. Also, collect information on their medical history, including any previous diagnoses or treatments related to dental or skeletal fluorosis. 
    \2 Ensure that the CT-scan and serum fluoride level measurements are conducted according to standard protocols and that the results are accurately recorded. 
\1 Store the collected data securely and ensure that it is only accessible to authorized members of the research team. Follow all relevant data protection and privacy regulations when handling patient data.
\1 Clean and preprocess the data as needed to prepare it for analysis. This may involve removing incomplete or inconsistent records, recoding variables, or aggregating data. 
\1 Conduct data analysis using appropriate statistical methods to address the research aims. Evaluate the diagnostic accuracy of CT-scans and serum fluoride levels in detecting skeletal fluorosis in the Somaliland population with dental fluorosis.

\end{outline}


\section{Anticipated Result}
%We anticipate that both CT-scans and serum fluoride levels will have high diagnostic accuracy in detecting skeletal fluorosis in the Somaliland population with dental fluorosis.
We anticipate that this study will provide valuable information on the prevalence of dental fluorosis and skeletal fluorosis in the Somaliland population. We also anticipate that our results will reveal an association between dental fluorosis and skeletal fluorosis in this population.

\section{Conclusion}
%This study will provide valuable information on the diagnostic accuracy of CT-scans and serum fluoride levels in detecting skeletal fluorosis in the Somaliland population with dental fluorosis. The results of this study may inform clinical practice and improve the diagnosis and management of skeletal fluorosis in this population.
This study will provide valuable insights into the prevalence and association of dental fluorosis and skeletal fluorosis in the Somaliland population. 
The results of this study may inform public health initiatives and improve the prevention, diagnosis, and management of fluorosis in this population.



\bibliographystyle{plainnat}
\bibliography{inline.bib}

\end{document}
